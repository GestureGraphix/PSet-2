\documentclass{report}

\input{preamble}
\input{macros}
\input{letterfonts}

\title{\Huge{Math 115 QR \\
PSet 2}}
\author{\huge{Alex Hernandez Juarez}}
\date{July 17 2024}

\begin{document}

\maketitle
\newpage% or \cleardoublepage
% \pdfbookmark[<level>]{<title>}{<dest>}
\pdfbookmark[section]{\contentsname}{toc}
\tableofcontents
\pagebreak

\chapter{}
\section{Problem 1: Gottlieb 27.1.4}

\qs{}{(a) A farmer has planted corn on a rectangular plot of land 800 meters by 1000 meters.
A straight stream runs alongside one of the long borders of the plot, and the farmer's
irrigation system is such that his yield decreases with the distance from the stream. Suppose his yield is given by
$f(x) = 50 - 0.3\sqrt{x}$ ears of corn per square meter, where $x$ is the distance from the stream in meters. What is the farmer's yield
from the plot? \\
(b) A second farmer plants his corn in a circular plot with radius 80 meters and he has
a centralized irrigation system located in the middle of his field. His yield drops
with the distance from the center of the field. Suppose his yield is also given by 
$f(x) = 50 - 0.3 \sqrt{x}$ ears of corn per square meter, this time $x$ being the distance 
from the center of the field. What is the farmers yield for this plot?
}

\sol{ 
	\\
	(a) \\
	$f(x) = 50 - 0.3 \sqrt{x}$ \\
	area of the $i^{th}$ slice $ = 1000 \Delta x$ \\
	ears of corn in the $i^{th}$ slice $ = 1000 \Delta x \cdot \left( 50 - 0.3 \sqrt{x} \right)$ \\
	total corn $ = \sum^{n}_{i=1} \left( 50 - 0.3\sqrt{x}\right) \cdot 1000 \Delta x$ \\
	total corn $ = \int_{0}^{800} \left( 50 - 0.3 \sqrt{x} \right) \cdot 1000  dx$
	\begin{center}
		\[ \int_{0}^{800} \left( 50 - 0.3 \sqrt{x} \right) \cdot 1000 dx  = 1000 \int_{0}^{800} \left( 50 - 0.3 \sqrt{x} \right) dx \]
		\[ 1000 \int_{0}^{800} \left( 50 - 0.3 \sqrt{x} \right) dx = \int_{0}^{800} 50000 - 300\sqrt{x} dx \]
		\[  \int_{0}^{800} 50000 - 300\sqrt{x} dx = 50000x - 200x\sqrt{x} \Big|^{800}_{0}\]
		\[ 50000x - 200x\sqrt{x} \Big|^{800}_{0} = 50000(800) - 200(800)\sqrt{800} \] 
		\[ 50000(800) - 200(800)\sqrt{800} = 35,474,516.6004 \] 
	\end{center}
	(b) \\
	$f(x) = 50 - 0.3 \sqrt{x}$ \\
	area of the $i^{th}$ slice $ = 2\pi x \Delta x$\\
	ears of corn in the $i^{th}$ slice = $ 2 \pi x \Delta x \left(50 - 0.3\sqrt{x}\right)$ \\
	total corn $ = \sum^{n}_{i=1} 2\pi \left(50x - 0.3x \sqrt{x} \right) \Delta x $ \\
	total corn $ = \int_{0}^{80} 2\pi \left(50x - 0.3x \sqrt{x} \right) dx $ \\
	\begin{center}
		\[\int_{0}^{80} 2\pi \left(50x - 0.3x \sqrt{x} \right) dx  = 2 \pi \int_{0}^{80} 50x - 0.3x \sqrt{x} dx \] 
		\[2 \pi \int_{0}^{80} 50x - 0.3x \sqrt{x} dx = \int_{0}^{80} 100 \pi x - 0.3 \cdot 2 \pi x^{\frac{3}{2}} dx \] 
		\[\int_{0}^{80} 100 \pi x - 0.3 \cdot 2 \pi x^{\frac{3}{2}} dx = 50 \pi x^{2} - \frac{3 \pi x^{2} \sqrt{x}}{25} \Big|_{0}^{80}\] 
		\[ 50 \pi x^{2} - \frac{3 \pi x^{2} \sqrt{x}}{25} \Big|_{0}^{80} =  983,729.4183\] 
	\end{center}
}

\section{Problem 2: Gottlieb 27.1.5}
\qs{}{Consider a box of cereal with raisins. The box is 5 centimeters deep, 25 centimeters
tall, and 16 centimeters wide. The raisins tend to fall toward the bottom; assume their density is 
given by $\rho (h) = \frac{4}{h+10}$ raisins per cubic centimeter, where $h$ is the above the 
bottom of the box. How many raisins are in the box?}
\sol{
	\\
	$ w = 5 $ \\
	$ h = 25 $ \\
	$ l = 16 $ \\
	area of the $i^{th} $ slice is $  16 \cdot 5 \Delta h = 80 \Delta x $ \\
	cereal in the $i^{th}$ slice is $ 80 \Delta x \frac{4}{h + 10}$ \\
	$ \int_{0}^{25} 80 \frac{4}{h + 10} dx$

	\begin{center}
	\[ \int_{0}^{25} 80 \frac{4}{h + 10} dh = \int_{0}^{25}\frac{320}{h + 10} dh \] 
	\[ \int_{0}^{25}\frac{320}{h + 10} dh = 320 \ln(x+10) \Big|_{0}^{25}\] 
	\[ 320 \ln(x+10) \Big|_{0}^{25} = 320ln(25+10) - 320 ln(0 + 10) \approx 400.884\] 
		
	\end{center} 
	}

\section{Problem 3: Gottlieb 27.1.14}
\qs{}{Liquid is being stored in a large spherical tank of radius 2 meters. The tank is completely
full and has been left standing for a long time. A mineral suspended in the liquid is
setting. Its density at a depth of $h$ meters from the top is given by $5h$ milligrams per
cubic meter. Determine the number of milligrams of the mineral contained in the tank.}
\sol{
	volume of $i^{th}$ slice $= \pi r^{2}  \Delta h$ \\
	$ r_{i} = \sqrt{4 - h_{i}^2}$  \\
	raisins in $i^{th}$ slice = $\pi \left( \sqrt{4 - h_{i}^{2}} \right)^{2} \cdot \Delta h \cdot  5h$ 
	\begin{center}
		\[ \int_{0}^{2} \pi \left(4 - h^{2}\right)  5h dh + \int_{0}^{2} \pi (4-h^{2}) 5(h+2) dh \]
		\[ 5 \pi \int_{0}^{2} 4x-x^{3} dx = 10 \pi x^{2} - \frac{5 \pi x^{4}}{4} \Big|_{0}^{2} = 20 \pi \]  
		\[ \int_{0}^{2} \pi (4-h^{2}) 5(h+2) dh = 10 \pi x^{2} + 40 \pi x - \frac{5 \pi x^{4} }{4} - \frac{10 \pi x^{3}}{3} \approx 230.3835\] 
		\[ \int_{0}^{2} \pi \left(4 - h^{2}\right)  5h dh + \int_{0}^{2} \pi (4-h^{2}) 5(h+2) dh \approx 293.215\]
	\end{center}
	
}

\section{Problem 4: Gottlieb 27.1.21 } 
\qs{}{A circus tent has cylindrical symmetry about its center pole. The height a distance of
$x$ feet from the center pole is given by $ h(x) = \frac{8}{1 + \frac{x^{2}}{16}}$ feet. What is 
the volume of the enclosed by the tent of radius 4.}
\sol {
	Volume of $i^{th}$ slice is: $\pi x h_{i} \Delta x$   
	\begin{center}
		\[ \int_{0}^{4} 2\pi x \frac{8}{1 + \frac{x^{2}}{16}} dx = 278.731\] 
	\end{center}
} 

\section{Problem 5: Stewart 6.4.8}
\qs{}{A spring has a natural length of 40 cm. \\
(a) If a 60-N force is required to keep the spring compressed 10 cm, how much
work is done during this compression?  \\
(b) How much work is required to compress the spring to a length of 25 cm?}

\sol{
	\\ 
	(a)
	\begin{center}
		\[ f(x) = kx \]
		\[ 60 N = k .1 \]
		\[ k = \frac{60}{0.1} = 600\]   
		\[ \int_{0}^{0.1} 600 x \space dx = \left( 300x^{2} \right) \big|_{0}^{10}\] 
		\[ \left( 300x^{2} \right) \big|_{0}^{10} = 300(0.1)^{2} = 3 \text{ J} \] 
	\end{center}
	(b)
	\begin{center}
		\[ \int_{0}^{0.15} 600 x \space dx = \left( 300x^{2} \right) \big|_{0}^{10}\] 
		\[ \left( 300x^{2} \right) \big|_{0}^{10} = 300(0.15)^{2} = 6.75 \text{ J} \] 
	\end{center}

}

\section{Problem 6: Stewart 6.4.12}
\qs{}{If 6 J of work is needed to stretch a spring from 10 cm to 12 cm and another 10 J is needed to stretch it from 12 cm
to 14 cm, what is the natural length of the spring?}
\sol{
	\begin{center}
		\[ \left( yx^{2} \right) \big|_{a}^{a + 2} = y(a + 2)^{2} - y(a)^{2} = 6 \text{ J} \] 
		\[ y(a + 2)^{2} - y(a)^{2} = 6 \] 
		\[ y(a + 4)^{2} - y(a+2)^{2} = 10 \] 
		\[ y(l -0.14)^{2} - y(l -.012)^{2} = 10 \] 
		\[ y\left(l - \frac{7}{50}\right)^{2} - y\left(l - \frac{3}{25}\right)^{2} = 10\] 
		\[ y\left(l - \frac{3}{25}\right)^{2} - y\left(l - \frac{1}{10}\right)^{2} = 6 \] 
		\[ y\left( \left(l - \frac{7}{50}\right)^{2} - \left(l - \frac{3}{25}\right)^{2} \right) = 10\] 
		\[ y\left(\left(l - \frac{3}{25}\right)^{2} - y\left(l - \frac{1}{10}\right)^{2}\right)= 6 \] 
		\[ y\left( \left(l - \frac{7}{50} - l + \frac{3}{25}\right) \left(l - \frac{7}{50} + l +\frac{3}{25}\right)\right) = 10\]  
		\[ y\left( \left(l - \frac{3}{25} - l + \frac{1}{10}\right) \left(l - \frac{3}{25} + l +\frac{1}{10}\right)\right) = 6 \]  
		\[ y\left(\left(-\frac{7}{50} + \frac{3}{25}\right) \left(l - \frac{7}{50} + l - \frac{3}{25}\right)\right) = 10\] 
		\[ y\left(\left(-\frac{3}{25} + \frac{1}{10}\right) \left(l - \frac{3}{25} + l - \frac{1}{10}\right)\right) = 6\] 
		\[ y \left( - \frac{1}{50} \left(2l - \frac{13}{50} \right)\right) = 10 \]
		\[ y \left( - \frac{1}{50} \left(2l - \frac{11}{50} \right)\right) = 6 \]
		\[ \frac{2l - \frac{13}{50}}{2l - \frac{11}{50}} = \frac{5}{3} \] 
		\[ \frac{100l - 13}{100l - 11} = \frac{5}{3}\]
		\[ 300l - 39 = 500 l - 55 \] 
		\[ -200l = -16 \] 
		\[ l = \frac{2}{25} \] 



	\end{center}

}

\section{Problem 7: Stewart 6.4.14}
\qs{}{A thick cable, 60 ft long and weighing 180 lb, hangs from a
winch on a crane. Compute in two different ways the work
done if the winch winds up 25 ft of the cable. \\
(a) Follow the method of Example 4 \\
(b) Write a function for the weight of the remaining cable
after $x$ feet has been wound up by the winch. Estimate
the amount of work done when the winch pulls up $\Delta x$ feet of cable }
\sol{
	(a) \\
	Weight per foot: $\frac{180}{60} = 3$ lb/ft 
	\begin{center}
		\[ W = \lim_{n \to \infty} \sum^{n}_{i=1} 3 x_{i} \Delta x \] 
		\[ \int_{0}^{25} 3x dx = \frac{3}{2}x^{2} \Big|_{0}^{25} \] 
		\[ \frac{3}{2}x^{2} \Big|_{0}^{25} = \frac{3}{2}(25)^{2} = 937.5\] 
	\end{center}
	(b) 
	\begin{center}
		\[ w = 180 - 3x \] 
	\end{center}
	What does that second part mean? 

}

\section{Problem 8: Stewart 6.4.24}
\qs{}{A tank is full of water. Find the work required to pump
the water out of the spout. $r = 3$m and spout = 1m}
\sol{
	\\
	Volume of $i^{th}$ slice is $\pi r^{2} dh$ \\
	Work is the distance traveled times the amount of water moved\\
	\begin{center}
		\[ r_{i}^{2} = 3^{2} - h_{i}^{2}\] 
		\[ r_{i} = \sqrt{3^{2} - h_{i}^{2}} \]
		\[ W = \pi \int_{-3}^{3} (4-y) 9 - h^{2} dh \] 
		\[ W = \left( 36 \pi - \frac{4x^{3}}{3} - \frac{9 \pi x^{2}}{2} + \frac{x^{4}}{4}\right) \Big|_{-3}^{3} \] 
		\[ W = 144 \pi \approx 451.389\] 
	\end{center}
}

\section{Problem 9: Stewart 6.4.26}
\qs{}{A tank is full of water. Find the work required to pump
the water out of the spout. In Exercises 25 and 26 use the fact that
water weighs 62.5 lb/ft$^{3}$. Traingular prism with $h = 6$, $w = 10$, and $ l = 12$}

\sol{ 
	The ratio of the legs is $\frac{12}{6}$ which is $2$ so the length will always be $2$ times the height.
	This is important as we are going to use the property of similar traignles. \\
	Height = $h_{i}$ \\
	Width = $10$ \\
	Length = $2h_{i}$ \\
	Volume of i$^{th}$ slice is $(10)(2h_{i}) \Delta h $ \\
	Amount of water in i$^{th}$ slice is $(10)(2h_{i})(62.5) \Delta h$
	\begin{center}
		\[ \text{Total work required to pump the water: } \int_{0}^{6} (62.5)(20h)(6-h) dh = 45000 \] 
	\end{center}

}
\end{document}
